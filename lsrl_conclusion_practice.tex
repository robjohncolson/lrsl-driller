\documentclass[12pt]{article}
\usepackage[utf8]{inputenc}
\usepackage{geometry}
\usepackage{amsmath}
\usepackage{tcolorbox}
\usepackage{array}
\usepackage{enumitem}

\geometry{letterpaper, margin=0.75in}
\setlength{\parindent}{0pt}
\setlength{\parskip}{10pt}

\begin{document}

\begin{center}
{\Large \textbf{Writing LSRL Conclusions}}\\[0.3cm]
{\large AP Statistics}\\[0.2cm]
Name: \underline{\hspace{6cm}} \hspace{1cm} Date: \underline{\hspace{3cm}}
\end{center}

\begin{tcolorbox}[colback=blue!5, colframe=blue!50!black, title=\textbf{The Three-Part Conclusion}]
For each scenario, write a complete conclusion with these three parts:

\textbf{1. Slope Interpretation:}\\
``For every additional [1 unit of x], the \underline{predicted} [y] increases/decreases by [b] [units].''

\textbf{2. Y-Intercept Interpretation:}\\
``When [x] = 0, the \underline{predicted} [y] is [a] [units].''\\
\textit{OR} ``The y-intercept has no meaningful interpretation because [explain why x = 0 is unrealistic].''

\textbf{3. Correlation Interpretation:}\\
``There is a [weak/moderate/strong] [positive/negative] \underline{linear} relationship between [x] and [y].''

\vspace{0.2cm}
\begin{center}
\renewcommand{\arraystretch}{1.2}
\begin{tabular}{|c|c|}
\hline
\textbf{$|r|$ value} & \textbf{Strength} \\
\hline
0.0 -- 0.3 & Weak \\
0.3 -- 0.7 & Moderate \\
0.7 -- 1.0 & Strong \\
\hline
\end{tabular}
\end{center}
\end{tcolorbox}

\hrule
\vspace{0.5cm}

\textbf{Scenario 1: Coffee and Productivity}

A company surveyed employees about their daily coffee consumption and work productivity. Let $x$ = cups of coffee per day and $y$ = tasks completed per day.

\begin{center}
\fbox{$\hat{y} = 12 + 3x$} \hspace{2cm} \fbox{$r = 0.72$}
\end{center}

\textbf{Slope Interpretation:}
\vspace{0.1cm}
\hrulefill

\vspace{0.4cm}
\hrulefill

\vspace{0.5cm}
\textbf{Y-Intercept Interpretation:}
\vspace{0.1cm}
\hrulefill

\vspace{0.4cm}
\hrulefill

\vspace{0.5cm}
\textbf{Correlation Interpretation:}
\vspace{0.1cm}
\hrulefill

\vspace{0.4cm}
\hrulefill

\newpage

\textbf{Scenario 2: Phone Screen Time and Sleep}

Researchers collected data from teenagers about their phone usage before bed. Let $x$ = hours of screen time after 8pm and $y$ = hours of sleep that night.

\begin{center}
\fbox{$\hat{y} = 8.5 - 0.75x$} \hspace{2cm} \fbox{$r = -0.64$}
\end{center}

\textbf{Slope Interpretation:}
\vspace{0.1cm}
\hrulefill

\vspace{0.4cm}
\hrulefill

\vspace{0.5cm}
\textbf{Y-Intercept Interpretation:}
\vspace{0.1cm}
\hrulefill

\vspace{0.4cm}
\hrulefill

\vspace{0.5cm}
\textbf{Correlation Interpretation:}
\vspace{0.1cm}
\hrulefill

\vspace{0.4cm}
\hrulefill

\vspace{1cm}
\hrule
\vspace{0.5cm}

\textbf{Scenario 3: Age and Marathon Time}

Data was collected from adult marathon runners. Let $x$ = age (in years) and $y$ = marathon finish time (in minutes).

\begin{center}
\fbox{$\hat{y} = 180 + 1.2x$} \hspace{2cm} \fbox{$r = 0.45$}
\end{center}

\textbf{Slope Interpretation:}
\vspace{0.1cm}
\hrulefill

\vspace{0.4cm}
\hrulefill

\vspace{0.5cm}
\textbf{Y-Intercept Interpretation:}
\vspace{0.1cm}
\hrulefill

\vspace{0.4cm}
\hrulefill

\vspace{0.5cm}
\textbf{Correlation Interpretation:}
\vspace{0.1cm}
\hrulefill

\vspace{0.4cm}
\hrulefill

\newpage

\textbf{Scenario 4: Study Time and Exam Scores}

A teacher tracked how long students studied for an exam. Let $x$ = hours spent studying and $y$ = exam score (out of 100).

\begin{center}
\fbox{$\hat{y} = 45 + 8x$} \hspace{2cm} \fbox{$r = 0.89$}
\end{center}

\textbf{Slope Interpretation:}
\vspace{0.1cm}
\hrulefill

\vspace{0.4cm}
\hrulefill

\vspace{0.5cm}
\textbf{Y-Intercept Interpretation:}
\vspace{0.1cm}
\hrulefill

\vspace{0.4cm}
\hrulefill

\vspace{0.5cm}
\textbf{Correlation Interpretation:}
\vspace{0.1cm}
\hrulefill

\vspace{0.4cm}
\hrulefill

\vspace{1cm}
\hrule
\vspace{0.5cm}

\textbf{Scenario 5: Temperature and Ice Cream Sales}

An ice cream shop tracked daily high temperature and sales. Let $x$ = high temperature ($^\circ$F) and $y$ = number of ice cream cones sold.

\begin{center}
\fbox{$\hat{y} = -150 + 4x$} \hspace{2cm} \fbox{$r = 0.83$}
\end{center}

\textbf{Slope Interpretation:}
\vspace{0.1cm}
\hrulefill

\vspace{0.4cm}
\hrulefill

\vspace{0.5cm}
\textbf{Y-Intercept Interpretation:}
\vspace{0.1cm}
\hrulefill

\vspace{0.4cm}
\hrulefill

\vspace{0.5cm}
\textbf{Correlation Interpretation:}
\vspace{0.1cm}
\hrulefill

\vspace{0.4cm}
\hrulefill

\newpage

\begin{center}
{\Large \textbf{Answer Key}}
\end{center}

\hrule
\vspace{0.3cm}

\textbf{Scenario 1: Coffee and Productivity} \hfill $\hat{y} = 12 + 3x$, \quad $r = 0.72$

\textbf{Slope:} For every additional cup of coffee per day, the predicted number of tasks completed increases by 3 tasks.

\textbf{Y-Intercept:} When an employee drinks 0 cups of coffee, the predicted number of tasks completed is 12 tasks. \textit{(This is meaningful---0 cups is realistic.)}

\textbf{Correlation:} There is a strong positive linear relationship between cups of coffee consumed and tasks completed.

\hrule
\vspace{0.3cm}

\textbf{Scenario 2: Phone Screen Time and Sleep} \hfill $\hat{y} = 8.5 - 0.75x$, \quad $r = -0.64$

\textbf{Slope:} For every additional hour of screen time after 8pm, the predicted hours of sleep decreases by 0.75 hours (or 45 minutes).

\textbf{Y-Intercept:} When a teenager has 0 hours of screen time after 8pm, the predicted sleep is 8.5 hours. \textit{(This is meaningful---0 hours of screen time is realistic.)}

\textbf{Correlation:} There is a moderate negative linear relationship between screen time after 8pm and hours of sleep.

\hrule
\vspace{0.3cm}

\textbf{Scenario 3: Age and Marathon Time} \hfill $\hat{y} = 180 + 1.2x$, \quad $r = 0.45$

\textbf{Slope:} For every additional year of age, the predicted marathon finish time increases by 1.2 minutes.

\textbf{Y-Intercept:} The y-intercept has no meaningful interpretation because an age of 0 years is not realistic for marathon runners. \textit{(Babies don't run marathons!)}

\textbf{Correlation:} There is a moderate positive linear relationship between age and marathon finish time.

\hrule
\vspace{0.3cm}

\textbf{Scenario 4: Study Time and Exam Scores} \hfill $\hat{y} = 45 + 8x$, \quad $r = 0.89$

\textbf{Slope:} For every additional hour spent studying, the predicted exam score increases by 8 points.

\textbf{Y-Intercept:} When a student studies for 0 hours, the predicted exam score is 45 points. \textit{(This is meaningful---some students don't study at all.)}

\textbf{Correlation:} There is a strong positive linear relationship between hours spent studying and exam score.

\hrule
\vspace{0.3cm}

\textbf{Scenario 5: Temperature and Ice Cream Sales} \hfill $\hat{y} = -150 + 4x$, \quad $r = 0.83$

\textbf{Slope:} For every additional degree Fahrenheit, the predicted number of ice cream cones sold increases by 4 cones.

\textbf{Y-Intercept:} The y-intercept has no meaningful interpretation because a temperature of 0$^\circ$F is outside the range of typical operating temperatures for an ice cream shop (and you can't sell negative cones).

\textbf{Correlation:} There is a strong positive linear relationship between high temperature and ice cream cones sold.

\end{document}
